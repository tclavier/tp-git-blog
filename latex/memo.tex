\documentclass[a4paper]{article}
\usepackage[utf8]{inputenc}
\usepackage[T1]{fontenc}
\usepackage{graphicx}
\usepackage{libertine}
\usepackage{tango}
%\usepackage[colorlinks=true]{hyperref}

\hypersetup{urlcolor=DarkChocolate}

\title{Mémo Git}
%\author{\scalebox{0.35}{ \includegraphics{logo_azae}}}
\author{Thomas Clavier}
\date{}

\setlength{\parindent}{0pt}
\newcounter{question}
\setcounter{question}{1}
\newcommand{\q}{
  \textcolor{DarkOrange}{\textbf{Question \thequestion : }}
  \addtocounter{question}{1}
  \newline
}

\begin{document}

\maketitle

\section*{Introduction}

L'ensemble de ce mémo est disponible sur \url{http://github.com/tclavier/tp-git-blog/} vous pouvez participer à son amélioration en proposant des <<pull request>>.

\section*{Présentation}

Git est un logiciel de gestion de versions décentralisé.

Ce mémo est un extrait du livre : \url{https://git-scm.com/book/fr}

\subsection*{Des objets}
Un dépôt Git peut être vu comme une collection d’objets liés entre eux. 
Chaque objet est identifié par une chaîne de 40 caractères hexadécimaux correspondant à la somme de contrôle de son contenu. 
Il y a 3 types d'objets : 
\begin{itemize}
\item blob : les données
\item tree : les arborescences
\item commit : une version du répertoire de travail
\end{itemize}

\begin{figure}
  \center
  \includegraphics{arbo_img}
  \caption{3 types d'objets}
  \label{objets}
\end{figure}
Comme illustré sur la figure \ref{objets} un <<commit>> est une image du répertoire de travail à un instant T. 
Le répertoire de travail est lui-même un objet de type <<arborescence>> qui va contenir des objets <<données>> et <<arborescences>>.

L'historique d'un dépôt Git c'est l'ensemble des versions du répertoire de travail. Pour identifier une version, Git s'appuie sur un objet de type <<commit>>. 
L'objet de type <<commit>> associe de nombreuses informations comme l'auteur, un message, une version de l'objet répertoire de travail mais aussi les <<commits> parents. 
Sur la figure \ref{graphs} on peut observer un historique de commit avec 3 branches.

\begin{figure}
  \center
  \includegraphics{graphs_img}
  \caption{Historiques de commits}
  \label{graphs}
\end{figure}
%TODO 3 schémas : graph de versions (branches) graphs de commits et tree + blob

\subsection*{Les espaces}
Git utilise 3 espaces différents pour manipuler toutes ces données.
\begin{itemize}
\item le répertoire de travail
\item l'index
\item l'historique
\end{itemize}

Le répertoire de travail présente dans un dossier de la machine, l'ensemble des fichiers du dépôt. C'est le point d'entrée de l'utilisateur.

L'index contient les données en préparation pour le commit

La tête (HEAD) de l'historique est un pointeur vers le commit qui sera utilisé comme prochain commit parent.

Sur la figure \ref{espaces} illustre les impactes de différentes commandes sur les 3 espaces de données.

\begin{figure}
  \centering
  \includegraphics[width=\textwidth]{espaces_img}
  \caption{Les 3 espaces de données}
  \label{espaces}
\end{figure}

\subsection*{À distance}
Avec les commandes push, fetch et pull, il est possible d'envoyer ou de recevoir la totalité de l'historique d'une branche vers un serveur distant. Ce qui permet de travailler à plusieurs sur le même projet.

Ces commandes sont illustrés sur la figure \ref{remote}.

\vspace{2mm}

\begin{figure}
  \center
  \includegraphics[width=\textwidth]{remote_img}
  \caption{Serveurs distants}
  \label{remote}
\end{figure}

\section*{Resources}
Pour avoir de l'aide sur git, il est possible de lancer "man git". Pour
avoir de l'aide sur la commande "git log" il est possible de lancer "man
git-log".

Le livre : \url{https://git-scm.com/book/fr} est une autre ressource importante.

\section*{Principales commandes}

Toutes les commandes git sont de la forme suivante : 
\begin{verbatim}
git action [options] argument
\end{verbatim}

\input includes/add.tex
\input includes/branch.tex
\input includes/checkout.tex
\input includes/cherry-pick.tex
\input includes/clean.tex
\input includes/clone.tex
\input includes/commit.tex
\input includes/config.tex
\input includes/diff.tex
\input includes/init.tex
\input includes/log.tex
\input includes/merge.tex
\input includes/mv.tex
\input includes/pull.tex
\input includes/push.tex
\input includes/rebase.tex
\input includes/remote.tex
\input includes/revert.tex
\input includes/rm.tex
\input includes/status.tex
\input includes/tag.tex


\section*{Fichiers particulier}

\subsection*{.gitignore}
Liste des fichiers ou répertoires que git doit ignorer



\end{document}

